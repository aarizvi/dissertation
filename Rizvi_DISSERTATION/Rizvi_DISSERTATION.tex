\documentclass[]{DissertateUSU}
\usepackage{lmodern}
\usepackage{amssymb,amsmath}
\usepackage{ifxetex,ifluatex}
\usepackage{fixltx2e} % provides \textsubscript
\ifnum 0\ifxetex 1\fi\ifluatex 1\fi=0 % if pdftex
  \usepackage[T1]{fontenc}
  \usepackage[utf8]{inputenc}
\else % if luatex or xelatex
  \ifxetex
    \usepackage{mathspec}
  \else
    \usepackage{fontspec}
  \fi
  \defaultfontfeatures{Ligatures=TeX,Scale=MatchLowercase}
\fi
% use upquote if available, for straight quotes in verbatim environments
\IfFileExists{upquote.sty}{\usepackage{upquote}}{}
% use microtype if available
\IfFileExists{microtype.sty}{%
\usepackage{microtype}
\UseMicrotypeSet[protrusion]{basicmath} % disable protrusion for tt fonts
}{}
\usepackage[top=1in,bottom=1in,right=1in,left=1.5in,includefoot]{geometry}
\usepackage{hyperref}
\hypersetup{unicode=true,
            pdftitle={Genetic Associations in Acute Leukemia Patients after Matched Unrelated Donor Allogeneic Hematopoetic Stem Cell Transplantation},
            pdfauthor={Abbas A Rizvi, B.S., M.S., M.Sc.},
            pdfborder={0 0 0},
            breaklinks=true}
\urlstyle{same}  % don't use monospace font for urls
\usepackage{color}
\usepackage{fancyvrb}
\newcommand{\VerbBar}{|}
\newcommand{\VERB}{\Verb[commandchars=\\\{\}]}
\DefineVerbatimEnvironment{Highlighting}{Verbatim}{commandchars=\\\{\}}
% Add ',fontsize=\small' for more characters per line
\usepackage{framed}
\definecolor{shadecolor}{RGB}{248,248,248}
\newenvironment{Shaded}{\begin{snugshade}}{\end{snugshade}}
\newcommand{\KeywordTok}[1]{\textcolor[rgb]{0.13,0.29,0.53}{\textbf{#1}}}
\newcommand{\DataTypeTok}[1]{\textcolor[rgb]{0.13,0.29,0.53}{#1}}
\newcommand{\DecValTok}[1]{\textcolor[rgb]{0.00,0.00,0.81}{#1}}
\newcommand{\BaseNTok}[1]{\textcolor[rgb]{0.00,0.00,0.81}{#1}}
\newcommand{\FloatTok}[1]{\textcolor[rgb]{0.00,0.00,0.81}{#1}}
\newcommand{\ConstantTok}[1]{\textcolor[rgb]{0.00,0.00,0.00}{#1}}
\newcommand{\CharTok}[1]{\textcolor[rgb]{0.31,0.60,0.02}{#1}}
\newcommand{\SpecialCharTok}[1]{\textcolor[rgb]{0.00,0.00,0.00}{#1}}
\newcommand{\StringTok}[1]{\textcolor[rgb]{0.31,0.60,0.02}{#1}}
\newcommand{\VerbatimStringTok}[1]{\textcolor[rgb]{0.31,0.60,0.02}{#1}}
\newcommand{\SpecialStringTok}[1]{\textcolor[rgb]{0.31,0.60,0.02}{#1}}
\newcommand{\ImportTok}[1]{#1}
\newcommand{\CommentTok}[1]{\textcolor[rgb]{0.56,0.35,0.01}{\textit{#1}}}
\newcommand{\DocumentationTok}[1]{\textcolor[rgb]{0.56,0.35,0.01}{\textbf{\textit{#1}}}}
\newcommand{\AnnotationTok}[1]{\textcolor[rgb]{0.56,0.35,0.01}{\textbf{\textit{#1}}}}
\newcommand{\CommentVarTok}[1]{\textcolor[rgb]{0.56,0.35,0.01}{\textbf{\textit{#1}}}}
\newcommand{\OtherTok}[1]{\textcolor[rgb]{0.56,0.35,0.01}{#1}}
\newcommand{\FunctionTok}[1]{\textcolor[rgb]{0.00,0.00,0.00}{#1}}
\newcommand{\VariableTok}[1]{\textcolor[rgb]{0.00,0.00,0.00}{#1}}
\newcommand{\ControlFlowTok}[1]{\textcolor[rgb]{0.13,0.29,0.53}{\textbf{#1}}}
\newcommand{\OperatorTok}[1]{\textcolor[rgb]{0.81,0.36,0.00}{\textbf{#1}}}
\newcommand{\BuiltInTok}[1]{#1}
\newcommand{\ExtensionTok}[1]{#1}
\newcommand{\PreprocessorTok}[1]{\textcolor[rgb]{0.56,0.35,0.01}{\textit{#1}}}
\newcommand{\AttributeTok}[1]{\textcolor[rgb]{0.77,0.63,0.00}{#1}}
\newcommand{\RegionMarkerTok}[1]{#1}
\newcommand{\InformationTok}[1]{\textcolor[rgb]{0.56,0.35,0.01}{\textbf{\textit{#1}}}}
\newcommand{\WarningTok}[1]{\textcolor[rgb]{0.56,0.35,0.01}{\textbf{\textit{#1}}}}
\newcommand{\AlertTok}[1]{\textcolor[rgb]{0.94,0.16,0.16}{#1}}
\newcommand{\ErrorTok}[1]{\textcolor[rgb]{0.64,0.00,0.00}{\textbf{#1}}}
\newcommand{\NormalTok}[1]{#1}
\usepackage{graphicx,grffile}
\makeatletter
\def\maxwidth{\ifdim\Gin@nat@width>\linewidth\linewidth\else\Gin@nat@width\fi}
\def\maxheight{\ifdim\Gin@nat@height>\textheight\textheight\else\Gin@nat@height\fi}
\makeatother
% Scale images if necessary, so that they will not overflow the page
% margins by default, and it is still possible to overwrite the defaults
% using explicit options in \includegraphics[width, height, ...]{}
\setkeys{Gin}{width=\maxwidth,height=\maxheight,keepaspectratio}
\IfFileExists{parskip.sty}{%
\usepackage{parskip}
}{% else
\setlength{\parindent}{0pt}
\setlength{\parskip}{6pt plus 2pt minus 1pt}
}
\setlength{\emergencystretch}{3em}  % prevent overfull lines
\providecommand{\tightlist}{%
  \setlength{\itemsep}{0pt}\setlength{\parskip}{0pt}}
\setcounter{secnumdepth}{0}
% Redefines (sub)paragraphs to behave more like sections
\ifx\paragraph\undefined\else
\let\oldparagraph\paragraph
\renewcommand{\paragraph}[1]{\oldparagraph{#1}\mbox{}}
\fi
\ifx\subparagraph\undefined\else
\let\oldsubparagraph\subparagraph
\renewcommand{\subparagraph}[1]{\oldsubparagraph{#1}\mbox{}}
\fi

%%% Use protect on footnotes to avoid problems with footnotes in titles
\let\rmarkdownfootnote\footnote%
\def\footnote{\protect\rmarkdownfootnote}

%%% Change title format to be more compact
\usepackage{titling}

% Create subtitle command for use in maketitle
\newcommand{\subtitle}[1]{
  \posttitle{
    \begin{center}\large#1\end{center}
    }
}

\setlength{\droptitle}{-2em}

  \title{Genetic Associations in Acute Leukemia Patients after Matched Unrelated
Donor Allogeneic Hematopoetic Stem Cell Transplantation}
    \pretitle{\vspace{\droptitle}\centering\huge}
  \posttitle{\par}
    \author{Abbas A Rizvi, B.S., M.S., M.Sc.}
    \preauthor{\centering\large\emph}
  \postauthor{\par}
    \date{}
    \predate{}\postdate{}
  
\newcommand{\yeardegree}{ 2019 } \newcommand{\degree}{ Pharmaceutical Sciences }
 \newcommand{\field}{ }
 \newcommand{\chairperson}{ }
 \newcommand{\committeeone}{ Lara E Sucheston-Campbell, MS, PhD, Adviser }
 \newcommand{\committeetwo}{ Guy Brock, PhD }
 \newcommand{\committeethree}{ Moray Campbell, PhD }
 \newcommand{\committeefour}{ Shili Lin, PhD }
 \newcommand{\gradschoolguy}{ The Ohio State University }
 % Tables
      \usepackage{booktabs}
      \usepackage{threeparttable}
      \usepackage{array}
      \newcolumntype{x}[1]{%
      >{\centering\arraybackslash}m{#1}}%
      \usepackage{placeins}
      \usepackage{chngcntr}
      \counterwithin{figure}{chapter}
      \counterwithin{table}{chapter}
      \usepackage[makeroom]{cancel}

\begin{document}
\maketitle

\pagenumbering{roman} \pagestyle{empty} \copyrightpage

\newpage

\pagestyle{fancy} \fancyhead[L]{Abstract} \fancyhead[R]{\thepage}
\fancyfoot[C]{} \chapter*{ABSTRACT}
\addcontentsline{toc}{section}{Abstract}

Here I will be writing an abstract that summarizes my dissertation
results

\newpage

\fancyhead[L]{Dedication} \fancyhead[R]{\thepage} \fancyfoot[C]{}
\chapter*{DEDICATION} \addcontentsline{toc}{section}{Dedication}

Dedicate it.

\newpage

\fancyhead[L]{Acknowledgments} \fancyhead[R]{\thepage} \fancyfoot[C]{}
\chapter*{ACKNOWLEDGEMENTS}
\addcontentsline{toc}{section}{Acknowledgments}

\begin{enumerate}
\def\labelenumi{\arabic{enumi}.}
\setcounter{enumi}{-1}
\tightlist
\item
  Lara
\item
  Ezgi
\item
  Barbara Foster
\item
  Martin Morgan
\item
  Seb
\item
  Moray
\item
  Friends
\item
  Bernie
\end{enumerate}

This work was supported by the NIH/NHLBI R01HL102278 and NIH/NCI
R03CA188733.

\begin{vita}

\dateitem{2008}{Williamsville North High School}

\dateitem{2012}{SUNY Fredonia}

\dateitem{2015}{University of Luxembourg}

\dateitem{2015}{SUNY at Buffalo, \\ 
	Roswell Park Cancer Institute, \\ 
	Graduate Division}

\dateitem{2016-present}{Graduate Research Associate,\\
			 Department of Pharmaceutics, \\
			 The Ohio State University.}


\begin{publist}

%% UPDATE FOR 2010:
%  Grad school only wants research publications, and it only wants those
%  research pubs that are actually published. Accepted or ``to appear''
%  publications don't count. If they look closely, they'll tell you to
%  remove any publications that aren't in print. Haivng said that, they
%  probably won't look that closely unless you put a really long list
%  here. You're tempting fate if you add instructional publications
%  though.

  \nobibliography{dissertation_references}
  \bibliographystyle{unsrt}

  \begin{enumerate}
    \item \bibentry{Rizvi_2018}
    \item \bibentry{Karaesmen_2017}
  \end{enumerate}


% \instructpubs
%
% \pubitem{B.~Simpson, ed.,
% \newblock ``Lab notes for Cow Science 101'', 1909.}

\end{publist}




\begin{fieldsstudy}

% The \majorfield* uses the unit specified in the \unit command used
% earlier in your document. If you want to use a different unit, use the
% second form shown here
\majorfield*
% \majorfield{Cow and Dairy Science}

%%
%% Note:  If there were only one field of study, the following list 
%%        would best be done using the following command:
%%
%%  \onestudy{Only Topic}{Only Professor}
%%

% \begin{studieslist}
% \studyitem{Topic 1}{Prof.\ Big Dude}
% \studyitem{Topic 2}{Prof.\ Other Dude}
% \studyitem{Topic 3}{Prof.\ Another Dude}
% \end{studieslist}

\end{fieldsstudy}

\end{vita}


\newpage

\fancyhead[L]{Table of Contents} \fancyhead[R]{\thepage} \fancyfoot[C]{}
\tableofcontents

\newpage

\fancyhead[L]{List of Tables} \fancyhead[R]{\thepage} \fancyfoot[C]{}
\listoftables

\newpage

\fancyhead[L]{List of Figures} \fancyhead[R]{\thepage} \fancyfoot[C]{}
\listoffigures

\newpage

\pagenumbering{arabic}

\newpage

\pagestyle{plain} \fancyhead[L]{} \fancyhead[R]{}
\fancyfoot[C]{\thepage} \chapter{Introduction}

\doublespacing

\section{Genetic Association Studies}\label{genetic-association-studies}

Genetic association studies are

\subsection{Candidate Gene Association
Studies}\label{candidate-gene-association-studies}

Candidate gene association studies (CGAS)

\subsection{Genome Wide Association
Studies}\label{genome-wide-association-studies}

Genome wide association studies (GWAS)

\section{DISCOVeRY-BMT}\label{discovery-bmt}

\subsection{Hematopoetic Stem Cell
Transplantation}\label{hematopoetic-stem-cell-transplantation}

Hematopoietic stem cell transplantation (HSCT) is an established
therapeutic procedure that is used as a potentially curative treatment
for life-threatening congenital or acquired blood disorders (malignant
or non-malignant). (Rizvi et al. 2018) HSCT involves the intravenous
infusion of autologous or allogeneic hematopoietic progenitor cells to
restore normal function in patients whose bone marrow is compromised.
Autologous HSCT involves self-donation of marrow stem cells, whereas
allogeneic HSCT is when stem cells are transferred from a HLA-matched
related donor (MRD) or a HLA-matched unrelated donor (MUD). Although a
matched sibling donor is preferred, only approximately 30\% of patients
who may benefit from HSCT have such a donor available. In the United
States, the number of allogeneic transplants yearly has dramatically
risen over the past decade, across all diseases. Patients with acute
myeloid leukemia (AML), acute lymphoblastic leukemia (ALL), or
myelodysplastic syndrome (MDS), represent the largest group treated with
allogeneic HSCT. While both patient care and matching has improved over
the past few decades almost half of all high-resolution 10/10 MUD-HSCT
recipients die within one-year post-transplant due to either their
disease or transplant-related causes.2 These trends also show
transplant-related causes are a larger contributor to mortality within
the first 100-days post-transplant and shift towards primary disease
after approximately six months post-transplant.2 Reducing TRM without
increasing risk of disease death and vice versa continue to represent a
substantial clinical challenge.

\subsection{Genotyping and Imputation}\label{genotyping-and-imputation}

All samples were genotyped using the Illumina Human OmniExpress BeadChip
and the Illumina HumanExome BeadChip (University of Southern California
Genomics Facility). Samples were assigned to plates to ensure the even
distribution of patient characteristics and potential confounding
variables using Optimal Sample Assignment Tool (OSAT), an R/Bioconductor
software package.74 Over 90\% of DISCOVeRY-BMT patients self-reported as
European American, Caucasian or White and thus replication and
validation analyses are performed on these recipient-donor pairs.
Stringent quality control was performed on both samples and SNPs within
this population. Population outliers were removed using EIGENSTRAT75
(n=73). Additional sample quality control removed samples with missing
call rate ≥ 2\% (n=54), sex mismatch (n=9), abnormal inbreeding
coefficients (n=20), and evidence of cryptic relatedness (n=17),
yielding 2106 and 777 donor-recipient pairs in cohorts 1 and 2,
respectively. Typed SNPs were removed if the call rate was
\textless{}98\%, there was deviation from Hardy-Weinberg equilibrium
proportions or discordance between duplicate samples was
\textgreater{}2\%. In total 637,655 and 632,823 SNPs from the
OmniExpress BeadChip were available for imputation in cohorts 1 and
cohort 2, respectively, using 1000 Genomes Project Phase 3. IMPUTE2
software was used for Imputation and QCTOOL was used to remove imputed
genotypes with info score \textless{}0.7, certainty \textless{}0.7 and a
minor allele frequency \textless{}0.00576,77. The recipient-donor
mismatch genome dosage calculations, described above, were done as the
absolute value of the recipient minus donor minor allele dosages.
Rs2066847 (SNP13) in NOD2/CARD15 was the only variant analyzed from the
Illumina HumanExome as it was not typed on the OmniExpress chip or
available following imputation.

\subsection{Statistical Models}\label{statistical-models}

Prior to genetic analyses, clinical covariates for inclusion in
genome-wide survival models were selected using bidirectional stepwise
Cox proportional hazard models of OS, PFS, TRM and DRM using R
statistical software.78 Cox proportional hazard models of OS, TRM and
DRM evaluated SNPs associated with time to death with all survivors
censored at 1 year post-BMT.79 PFS was defined as the time to disease
progression or death. Deaths from TRM and DRM were treated as competing
risks and analyzed accordingly.80 SNP models for OS adjusted for
recipient age, disease status (early/intermediate or advanced), and
graft source (blood or marrow); PFS and DRM SNP models adjusted for
recipient age and disease status; TRM SNP models adjusted for recipient
age, graft source and body mass index (underweight/normal, overweight,
or obese). Dosage data accounting for the probability of each genotype
were used in all analyses of imputed data. Effect size estimates and
standard errors from DISCOVeRY-BMT Cohorts 1 and 2 were compared and
combined using a fixed-effects inverse variance meta-analyses in METAL.
For each SNP, heterogeneity of effect size estimates between cohorts 1
and 2 was assessed using p-values from significance tests of
heterogeneity (phet) and I2.81 Variants with phet\textless{}0.05 and
I2\textgreater{}50 were meta-analyzed with a random effects models using
meta in R.82

\subsubsection{Cox Proportional Hazards
Model}\label{cox-proportional-hazards-model}

\subsubsection{Power Calculations}\label{power-calculations}

We conducted the first adequately powered evaluation of these candidate
SNP and gene hypotheses using typed and imputed data from an existing
genome-wide association study (GWAS) named Determining the Influence of
Susceptibility COnveying Variants Related to one-Year mortality after
BMT (DISCOVeRY-BMT) to replicate or validate these published
associations.\(^{71-72}\) In addition, we leveraged the available
genome-wide data from DISCOVeRY-BMT and measured the aggregate
association of all SNPs in the candidate genes with survival outcomes to
determine how many of these candidate genes play a significant role in
survival after transplant. Lastly, using publically available data, we
characterized the potential functionality of each candidate SNP in
relation to the gene of interest.

\section{Literature Review}\label{literature-review}

An extensive literature search of PubMed was performed using to identify
peer-reviewed scientific studies (published on or before December 30,
2016) that reported non-HLA genetic polymorphisms associated with
survival outcomes after allogeneic BMT, including disease-related
mortality (DRM), progression-free survival (PFS), transplant-related
mortality (TRM) and/or overall survival (OS).1-70 The PubMed search
terms, filtering approach and link to all articles described herein are
provided in the Supplemental Methods.

For over a decade, researchers have conducted candidate gene association
studies of patient survival outcomes after allogeneic blood or marrow
transplantation (BMT). The intent of these studies was to identify
genetic variants outside of the human leukocyte antigen (HLA) region
that would increase knowledge about clinical management or serve as a
potential target for novel therapeutics. \(^{1-70}\)

The majority of these studies tested for associations in small datasets,
ranging from a few dozen to a few hundred patients and donors, included
heterogeneous diseases spanning benign to malignant hematological
diseases, related and/or unrelated donors with various degrees of
HLA-matching and patients treated across multiple decades, from the
1980s through early 2000s.

\section{Reproducing results of previous
studies}\label{reproducing-results-of-previous-studies}

Results from genetic association studies should be reproduced in
independent samples in order to confirm findings.83 Researchers have
defined two distinctive terms to describe the reproducibility based on
differences between the original study population and the confirmation
studies: replication and validation.84 Replication is defined as the
original and confirmation studies both having similar inclusion criteria
(including the same ethnic/ancestral population) so that any differences
between the study populations can be attributed to random variation.84
Validation is defined as the original and confirmation study populations
having different inclusion criteria (including different
ethnic/ancestral populations) so that any differences between the
original and confirmation study could be due to systematic variation.84
Thus, replication analyses were conducted when the original study
included HLA-matched unrelated donor BMTs in patients of European
ancestry. Validation analyses were performed on studies of leukemia
patients of non-European ancestry, patient populations who received a
BMT from a matched related donor, or patient populations that were mixed
between those who received a BMT from related and unrelated donor. For
studies of outcomes involving multiple hematologic malignancies, the
entire DISCOVeRY-BMT study population was analyzed. If the original
study population was specified as AML, ALL and/or MDS, the same disease
inclusion criteria were applied so that the replication/validation study
population aligned with that of the original study population.

\section{Our study}\label{our-study}

\subsection{Gene-Based Association
Testing}\label{gene-based-association-testing}

VErsatile Gene-based Association Study 2 (VEGAS2) software was used for
gene-based association testing.85 VEGAS2 uses 106 Monte Carlo
simulations to test the global significance of an association for sets
of SNPs in defined genomic regions. VEGAS2 reports a gene-based P-value
for each gene determined using individual SNP association P-values.
Directional effects are not incorporated into analyses; thus, all SNPs
can be aggregated without dampening an association signal. For the
gene-based replication or validation analyses, the P-values from typed
and imputed SNPs in DISCOVeRY-BMT (+/- a 10kb flanking region)
meta-analyses of OS, PFS, TRM and DRM were used as input into the VEGAS2
software. Gene-based P-values were calculated for donor, recipient, and
R-D mismatch analyses of the full cohort (ALL, AML and MDS patients) or
homogenous disease subgroups (ALL or AML or MDS patients) corresponding
to the analyses performed in the original studies.

\subsection{Functional Annotation}\label{functional-annotation}

RegulomeDB,86 Blood expression quantitative trait loci (eQTL) Browser,87
and Variant Effect Predictor (VEP)88 were used to provide functional
annotation of the candidate SNPs. For each database, the raw data
scores, P-values and annotations, respectively were downloaded from each
website and assigned to each SNP in our list. RegulomeDB scores are
categorized as follows: 1a-1f are likely to affect transcription factor
binding and linked to expression of a gene target; 2a-2c are likely to
affect transcription factor binding; 3a-3b are less likely to affect
transcription factor binding, and \textgreater{} 3 has minimal binding
evidence. A RegulomeDB score is assigned based on the level and evidence
of functional modification attributable to the SNP 86,89 in multiple
cell lines from a range of tissues, with scores from 1 to 7, with 1
having the highest functional effect, supported by experimental evidence
and 7 having no modifying effect.89 RegulomeDB database derives these
annotations using the publically available data sets from Gene
Expression Omnibus (GEO), the Encyclopedia of DNA elements (ENCODE)
project and the Roadmap Epigenome Consortium. The Blood eQTL data are
derived from a study of correlations between genetic variants and gene
expression in over 5000 patients, with replication in almost 3000
individuals. Herein, we consider only cis-eQTLs, defined as \textless{}
250KB distance between the SNP chromosomal position and the probe
midpoint for gene expression. VEP was used to determine the hypothetical
functional importance of missense and nonsense variants based on SIFT90,
Mutation Taster91 and PolyPhen-2.92

\section{Results}\label{results}

\subsection{DISCOVeRY-BMT Patient
Characteristics}\label{discovery-bmt-patient-characteristics}

DISCOVeRY-BMT cohorts 1 and 2 include mostly 10/10 HLA-matched unrelated
donors, with 281 8/8 HLA-matched donor-recipient pairs in cohort 2; all
patients are of European continental ancestry. Cohorts do not differ by
intensity of conditioning regimen, recipient or donor sex proportions,
KPS/LPS scores. However, cohort 1 includes more ALL patients whereas
cohort 2 includes more recipients with MDS. AML disease status also
differs between cohorts at p\textless{}0.01 (Table 1).

\subsection{Candidate Gene Studies of Survival
Outcomes}\label{candidate-gene-studies-of-survival-outcomes}

The literature search identified 70 publications that studied a total
458 SNPs and 2 multi-allelic polymorphisms in 171 genes (Figure 1, Table
S1). Studies included patients who received a transplant from an
HLA-matched unrelated donor (19 articles), an HLA-matched related donor
(23 articles), or both (28 articles) (Table S1). Study populations
included patients and donors of European ancestry (53 articles), Asian
ancestry (15 articles), or mixed genomic ancestry (2 articles) (Table
S1).

A total of 14 articles assessed genetic variation in HLA matched
unrelated donor (URD) BMT patients of European ancestry, but only 7 of
these articles reported significant associations (P\textless{}0.05 or an
author specified significance threshold) and thus comprise our
replication study (Table S2, Table S3). A total of 56 articles tested
associations in either a combination of related and unrelated donors
(RD-URD), only related donors (RD) and/or in non-European populations;
39 of these 56 articles reported at least one significant SNP
association with survival outcome and we attempted to validate the
significant findings from these 39 articles (Table S2, Table S4).

\subsection{Replication}\label{replication}

DISCOVeRY-BMT cohorts were used to replicate published studies of
European American acute leukemia or MDS patients treated with an
unrelated donor BMT.1-14 Of the 7 articles whose findings we attempted
to replicate, 2 articles tested multi-allelic models in NOD2/CARD155 and
CCR56; 5 articles tested single SNP associations in TGFB11, CD2743,
CD403, TNFSF43, HMGB14,IL1A7, IL1B7, and NOD2/CARD152 (Table 2, Figure
2, Table S3).1-7

The two NOD2/CARD15 associations were based on a three-variant R-D pair
model {[}rs2066844 (SNP8), rs2066845 (SNP12) and rs2066847 (SNP13){]}
and single SNP associations with SNP13.27 The null type is when the R-D
pair are homozygous common allele for all three SNPs and the effect
allele combination is the presence of 1 or more minor alleles at any of
the three SNPs within the R-D pair. In a study of 196 patients who
received an unrelated donor BMT for AML or ALL, the NOD2/CARD15
multi-SNP model was significantly associated with OS (RR: 1.6, 95\% CI
1.1-2.4, P=0.02) and TRM (RR: 1.6, 95\% CI 1.1-2.4, P=0.02).5 However,
in the DISCOVeRY-BMT AML and ALL patients (n=1597) treated with an
unrelated donor BMT, there was no association with OS (HR: 1.03, 95\% CI
0.9-1.2, P=0.72) or TRM (HR: 1.1, 95\% CI 0.8-1.4, P=0.6, Figure 2,
Table S3). In a study of 342 unrelated donor genotypes matched with AML
or ALL patients, rs2066847 (SNP13) alone significantly increased risk of
TRM and OS approximately 3-fold (P=0.001) and 2.5 (P=0.001),
respectively2, however DISCOVeRY-BMT donor genotypes, did not associate
with either TRM (HR: 1.17, 95\% CI 0.78-1.74, P=0.45) or OS (HR: 0.98,
95\% CI 0.73-1.31, P=0.89, in ALL or AML patients (Table 2, Figure 2,
Table S3).

One of the largest candidate gene studies (N=1370) showed significant
associations between PFS and recipient CCR5 H1/H1 genotype (n=163), as
well as with author defined genotype risk subgroups and OS.6 In
DISCOVeRY-BMT, neither the CCR5 H1/H1 genotype (n=294) nor the genotype
risk groups defined by H1/H16 status were significantly associated with
PFS or OS (Figure 2, Table S3). The genotype risk groups tested by the
authors were substantially smaller than the full cohort (Table 2). In
DISCOVeRY-BMT these subgroups were approximately twice as large as those
in the original study and adequately powered to detect these
associations. Attempts to replicate single SNP associations in TNFSF4,3
TGFB1,1 HMGB15, IL1A7, and IL1B7 also failed (Table 2, Figure 2, Table
S3).

\subsection{Validation}\label{validation}

We attempted to validate 36 polymorphisms in 26 genes from 39 candidate
gene articles (Table S2, Table S4),15-52 including: ABCB129,32, CD1442,
CTLA428,40,43-46,51, CYP2C1938, DAAM252, EP30036, ESR117, GSTA219,
GZMB24, ICAM148, IL23R20,22, IL615-17, IRF337, KLRK123, LIG348,
MTHFR31,35,41, MUTYH48, NOD2/CARD1525,27,30,33,50, NOS130, P2RX734,
TDG48, TIRAP17, TLR442, TYMP26, and VDR18,21,39,47. These studies
reported significant genetic associations with survival after transplant
in patients who received a HLA-matched related donor BMT (19 articles)
or had a study population including HLA-matched related and unrelated
donor BMT patients, without stratification of results (17 articles). We
also attempted to validate survival associations seen in non-European
leukemia patients who received an unrelated donor BMT (3 articles). We
present results of variants reported significant in at least two
separate publications in Table 3 and Figure 3.

Our validation analyses identified only one variant associated at
P\textless{}0.05. Donor variation in rs1800795 (IL-6) associated with OS
(HR: 1.11, 95\% CI 1.0-1.2, P=0.02) (Figure 3, Table S4). This SNP
association was initially reported in a single study by Balavarca et
al., 2015, (HR: 1.29, 95\% CI 1.07-1.55, P=0.007) in patients with acute
leukemia, CML, or lymphoma treated with a matched related or unrelated
donor BMT (n=743).

SNPs within NOD2/CARD15 were the most frequently studied and reported of
all candidate gene association studies in our validation set (Table S2).
NOD2/CARD15 is a susceptibility gene for inflammatory bowel disease and
may be involved in Crohn's disease.27 We attempted to validate studies
that reported an association of NOD2/CARD15 and survival outcomes in
HLA-matched related and unrelated donor BMT patients27,30,33 or
HLA-matched related donor BMT patients.25,50 Three studies reported
significant findings between the presence of the NOD2/CARD15 multi-SNP
polymorphism in either donor or recipient with TRM27,50 or PFS,25
however this did not validate in the DISCOVeRY-BMT cohorts (Figure 3,
Table 3). There was also no significant association of the single
variant rs2066842 in related/unrelated donors with PFS,30 or the single
variant rs2066847 (SNP13) in recipients of related/unrelated donor BMTs
with TRM (Figure 3, Table 3)33 in the DISCOVeRY-BMT cohorts.

Due to its known functions and perceived implications in transplant
biology,43 associations with multiple SNPs in CTLA4 have been tested in
numerous transplant populations (Table S2), with 4 CTLA4 SNPs
(rs3087243, rs231775, rs4553808, rs5742909) reported as significantly
associated with survival after related or unrelated donor allogeneic BMT
in acute leukemias, CML, lymphomas, MDS, and other hematological
disorders (Table 3). Attempts to validate CTLA4 SNPs with DRM, PFS, OS,
and TRM were unsuccessful in the DISCOVeRY-BMT cohorts (Table 3, Figure
3, Table S4).

The remaining results of the 25 additional candidate genes containing
SNPs that were tested in the DISCOVeRY-BMT cohorts are summarized in
Tables S4 and 3 as well as Figure 3; no SNP associations were found at
P\textless{}0.05. Importantly, the P-value distribution of the single
SNP associations showed no deviation from the null expectation with 95\%
confidence intervals (Figure S2), suggesting we cannot reject the null
hypothesis of no association with survival outcome.

\subsection{Gene based replication and validation of previous
studies}\label{gene-based-replication-and-validation-of-previous-studies}

The reviewed candidate gene studies first selected genes based on their
hypothesized or known function, and subsequently selected variants
within that gene for single SNP or haplotype testing. Thus, while SNPs
and haplotypes were tested individually for association, the hypotheses
from the literature can be considered gene-based. The density of typed
and imputed markers in the DISCOVeRY-BMT recipients and donors allows us
to measure the aggregate effect of all SNPs within each candidate gene
on survival. Genes were selected for testing from the same literature
summarized above for the replication and validation SNP and haplotype
analyses. VEGAS2 gene-based testing did not reveal any associations at
P\textless{}0.05 with any of the survival outcomes in either the
replication or validation groups (Table S5).

\subsection{Candidate polymorphism
annotation}\label{candidate-polymorphism-annotation}

Candidate gene SNPs were analyzed using the RegulomeDB,86 VEP88 and
Blood eQTL Browser87 databases to assess their functional
characteristics and better understand their biological framework. Eighty
percent of previously reported SNPs had RegulomeDB scores greater than 3
(Figure 4, Table S6), indicating that these SNPs have minimal to no
effect on modifying transcription. This distribution aligns with the
overall distribution of SNPs in the genome, thus the candidate SNPS are
not enriched for their impact on gene expression or transcription factor
binding. Our replication and validation analyses includes 2 protein
coding variants, VEP shows that only, rs2066845 (SNP12) in NOD2/CARD15,
is predicted to be damaging and disease causing.

The Blood eQTL browser determines if candidate SNPs have a significant
role in cis gene expression of the candidate gene. Of the 171 genes
included in our literature search results, 52\% have at least one
significant cis-eQTL at a probe-level false discovery rate (FDR)
\textless{} 0.05. On a genome-wide level, approximately 44\% of genes
have blood cis-eQTLs (FDR P\textless{}0.05). However, despite over half
of the candidate genes having blood cis-eQTLS, only 13\% of the
candidate SNPs reported in these articles are blood cis-eQTLs. Thus,
while blood eQTLs have been identified in these genes, they were not
genotyped and analyzed in these candidate gene studies. Furthermore,
almost half of the eQTLs in the candidate gene studies are correlated
with expression that is not the candidate gene but rather a nearby gene.
For example, rs7975232 (VDR) is an eQTL for SLC48A1 while the CTLA4 SNPs
are actually eQTLs for CD28. The remaining eQTLs were correlated with
expression of the candidate gene of interest, but in most cases, were
also significant eQTLs for several other nearby genes (Table S6).

\section{Discussion}\label{discussion}

Our study aimed to replicate or validate all previous genetic
association studies that investigated the non-HLA genetic effects on
allogeneic BMT survival. Since previous studies selected SNPs in
candidate genes, we conducted both single SNP and gene-based analyses to
determine the aggregated SNP associations within candidate genes while
still accounting for dependence between signals due to LD.

The only association with P\textless{}0.05 in our replication and
validation analyses using DISCOVeRY-BMT was the donor SNP rs1800795 in
IL-6 with OS. As reported,18 the rationale for studying this SNP was
based on the immunological function of IL-6 and two prior findings
showing that it was associated with GvHD93, and response to chronic
Hepatitis C virus therapy.94 We found no evidence of association at
P\textless{}0.05 between donor SNP rs1800795 with death due to either
GvHD or infection in the DISCOVeRY-BMT cohort (data not shown).
Furthermore, rs1800795 is located in the intronic region of IL-6, has no
effect on IL-6 expression or levels,95 but rather is an eQTL for two
other nearby genes.95,96

In addition to exploring this IL-6 association further we felt the
validation of the CCR5 associations of H1/H1 genotype with outcome
required additional efforts, as these associations were found in the
largest study we attempted to validate, samples were also from CIBMTR
(earlier years than our study population) and unlike many of the other
studies survival effects only started to appear approximately two years
post-transplant. Analyses outlined in Table 2 were performed without
censor at 1 year for overall survival (median survival time 13.7 months,
range \textless{}1 month-125.6 months) and progression free survival
(median time 11.1 months, range \textless{}1 month-125.6 months). There
were no genotype associations with either outcome at P\textless{}0.10.

Another frequently studied gene, CTLA4, highlights the heterogeneity
specific to studies of genetic variation in transplant and perhaps helps
explain why we did not replicate or validate associations. rs5742909 in
CTLA4 was tested for association with various survival outcomes after
transplant in 6 independent studies of HLA matched-related
donor-recipient pairs. In donors, the variant was found to be associated
with DRM in one small study (N=120), this was the only study that tested
donor genotype with DRM. Likewise, 1 out of 9 papers testing the
association of rs231775 with survival outcomes measured the association
of PFS with recipient rs231774 in 164 recipients (P=0.025). Despite the
frequency with which these two CTLA4 variants were studied, for both
SNP-outcome combinations DISCOVeRY-BMT is the only validation attempt.
These SNPs are like those of many candidate gene hypotheses, in that
they have not been tested in the same genome for the same outcome in
similar populations, and if they have the N is small (Table S1).

Our inability to replicate or validate previous candidate gene
associations could also be due to differences in inclusion criteria with
respect to disease, donor relation, or to differences in our endpoint of
1-year survival versus longer-term survival. The previous genetic
associations were hypothesized to be independent of underlying
hematologic disease, therefore we would expect to replicate or validate
these associations in a homogeneous patient population such as
DISCOVeRY-BMT. When possible we aligned our study population to the
original candidate gene study (i.e.~restricted to AML patients only).
While DISCOVeRY-BMT focused on early 1-year survival, which may have
different genetic contributions than later survival, many of the
survival curves in the significant candidate gene articles show
separation by genotype well before 1-year post-transplant, thus the
significant published variants do not appear to correlated with only
longer-term survival.

The large sample size of the DISCOVeRY-BMT provides adequate statistical
power to attempt replication and validation of previously published
candidate gene analyses71, however we did not reproduce these findings,
similar to two other recent studies attempting to replicate previous
candidate gene associations with GvHD after BMT.73,97 Other reports have
also concluded that a substantial amount of the published candidate gene
literature has presented false positive associations.98

Confirming genetic association studies is vital to identify true
positive genetic variants that may contribute to complex phenotypes.
False associations lead to wasted time, energy and money in pursuit of
confirmatory studies and could harm patients by delaying clinical
discovery or by applying clinical studies too quickly without
replication. Annotation of the previously reported SNP associations
using publically available data show that few variants are functional;
only one SNP is predicted to be damaging or deleterious, a small
proportion of SNPs are correlated with gene expression, and an even
smaller number are cis-eQTLs for the target gene of interest. Thus,
while we did not replicate or validate these associations, the SNPs
selected are not linked to functional annotation nor are they clearly
related to the candidate genes. This underscores a fundamental problem
with candidate gene studies which are hostage to the state of scientific
knowledge at the time. Adequately powered testing of genetic
associations with transplant outcomes remains critical to discovery and
replication of genetic associations with the ultimate goal of improving
patient outcomes.

\FloatBarrier

\newpage

\pagestyle{plain} \fancyhead[L]{} \fancyhead[R]{}
\fancyfoot[C]{\thepage}

\chapter{gwasurvivr}

Genome-wide association studies (GWAS) are population-level experiments
that investigate genetic variation in individuals to observe single
nucleotide polymorphism (SNPs) associations with a phenotype. Genetic
variants tested for association are genotyped on an array and imputed
from a reference panel of sequenced genomes, e.g.~1000 Genomes Project
or Haplotype Reference Consortium (HRC).

Imputed SNPs can be tested for association with binary outcomes
(cases/controls) dnd quantitative outcomes (e.g., height) using a range
of available software packages, including SNPTEST (Marchini, et al.,
2007) or PLINK (Purcell, et al., 2007). However, existing software
options for performing survival analyses, genipe (Lemieux Perreault, et
al., 2016), SurvivalGWAS\_SV (Syed, et al., 2017), and GWASTools
(Gogarten, et al., 2012) either require user interaction with raw
output, were not initially designed for survival and/or have long run
times. For these reasons, we developed an R/Bioconductor package,
gwasurvivr, for genome wide survival analyses of imputed data in
multiple file formats with flexible analysis and output options.

\section{Data Structure}\label{data-structure}

Gwasurvivr can analyze data in IMPUTE2 format (Howie, et al., 2009), in
VCF files derived from Michigan (Das, et al., 2016) or Sanger imputation
servers (McCarthy, et al., 2016), and directly genotyped PLINK format
(Purcell, et al., 2007). Data from each are prepared in gwasurvivr by
leveraging existing Bioconductor packages GWASTools (Gogarten, et al.,
2012) or VariantAnnotation (Obenchain, et al., 2014) depending on the
imputation file format.

IMPUTE2 Format: IMPUTE2 (Howie, et al., 2009) format is a standard
genotype (.gen) file which store genotype probabilities (GP). We
utilized GWASTools in R to compress files into genomic data structure
(GDS) format (Gogarten, et al., 2012). This allows for efficient,
iterative access to subsets of the data, while simultaneously converting
GP into dosages (DS) for use in survival analyses.

VCF Format: VCF files generated from these Michigan or Sanger servers
include a DS field and server-specific meta-fields (INFO score
{[}Sanger{]} or \(r^2\) {[}Michigan{]}, as well as reference panel
allele frequencies) that are iteratively read in by VariantAnnotation
(Obenchain, et al., 2014).

PLINK Format: Plink bed files contain genotype information encoded in
binary format. Fam and bim files include the information of phenotype
and marker location, respectively (Purcell, et al., 2007).

gwasurvivr implements a Cox proportional hazards regression model (Cox,
1992) to test each SNP with an outcome with options for including
covariates and/or SNP-covariate interactions. To decrease the number of
iterations needed for convergence when optimizing the parameter
estimates in the Cox model we modified the R package survival (Therneau
and Grambsch, 2000). Covariates in the model are first fit without the
SNP, and those parameter estimates are used as initial points for
analyses with each SNP. If no additional covariates are added to the
model, the parameter estimation optimization begins with null initial
value. (Supplementary Figure 1).

\section{Survival Analysis}\label{survival-analysis}

Survival analyses are run using genetic data in either VCF or IMPUTE2
(Howie, et al., 2009) formats and a phenotype file, which contains
survival time, survival status and additional covariates, both files are
indexed by sample ID. In addition to genomic data, the VCF files contain
both sample IDs and imputation quality metrics (INFO score or \(r^2\)),
while IMPUTE2 (Howie, et al., 2009) come in separate files (.gen,
.sample, and .info). Gwasurvivr functions for IMPUTE2 (impute2CoxSurv or
gdsCoxSurv) and VCF (michiganCoxSurv or sangerCoxSurv) include arguments
for the survival model (event of interest, time to event, and
covariates) and arguments for quality control that filter on minor
allele frequency (MAF) or imputation quality (michiganCoxSurv and
sangerCoxSurv only). INFO score filtering using impute2CoxSurv can be
performed by accessing the .info file from IMPUTE2 results and
subsequently providing the list of SNPs to `exclude.snps' argument to
gwasurvivr. Users can also provide a list of sample IDs for gwasurvivr
to internally subset the data. gwasurvivr outputs two files: (1)
.snps\_removed file, listing all SNPs that failed QC parameters and (2)
.coxph file with the results from the analyses, including parameter
estimates, p-values, MAF, the number of events and total sample N for
each SNP. gwasurvivr also allows the number of cores used during
computation on Windows and Linux to be specified. Users can keep
compressed GDS files after the initial run by setting keepGDS argument
to TRUE when analyzing IMPUTE2 data (Howie, et al., 2009). On successive
runs, gdsCoxSurv can then be used instead of impute2CoxSurv to avoid
compressing the data on each GWAS run.

\section{Simulations and
Benchmarking}\label{simulations-and-benchmarking}

Computational runtimes for gwasurvivr were benchmarked against existing
software comparing varying sample sizes and SNP numbers, with 4, 8 or 12
covariates and for a single chromosome with 15,000-25,000 individuals.
In addition, we evaluated time for gwasurvivr for a GWAS
(\textasciitilde{}6 million SNPS) for 3000, 6000 and 9000 samples. All
benchmarking experiments were performed using IMPUTE2 format (comparison
packages do not take VCF from either imputation servers).

Descriptions of simulated genotype and phenotype data are in the
Supplementary Data.

\section{Results}\label{results-1}

gwasurvivr was faster than genipe (Lemieux Perreault, et al., 2016),
SurvivalGWAS\_SV (Syed, et al., 2017), and GWASTools (Gogarten, et al.,
2012) for 100,000 SNPs at N=100, and 5000, with the exception of
SurvivalGWAS\_SV at N=1000 (Figure 1A). Similarly, increasing the number
of covariates for gwasurvivr has minimal effects on runtime versus other
software (Figure 1B). Gwasurvivr computes for large sample sizes,
however, compression time increases with increasing sample size, and
likely will be limited by available RAM on a machine or cluster (Figure
1C). The keepGDS argument helps address this and results in reduced run
times (Figures 1C and 1D), i.e. \textless{} 3 hours for a GWAS of
N=9,000. A \textasciitilde{}6 million SNP GWAS can be run in \textless{}
10 hours for 9000 samples when using separately scheduled jobs on a
supercomputer (Figure 1D). However, gwasurvivr overcomes memory
limitations often attributed to R by processing subsets of the entire
data, and thus it is possible to conduct genome-wide survival analyses
on a typical laptop computer.

gwasurvivr is a fast, efficient, and flexible program well suited for
multi-core processors and easily run in a computing cluster environment.

\FloatBarrier

\newpage

\pagestyle{plain} \fancyhead[L]{} \fancyhead[R]{}
\fancyfoot[C]{\thepage}

\chapter{Benchmarking survival model software}

\texttt{gwasurvivr} is an R package that can be used to conduct survival
analysis (Cox proportional hazards model) on imputed GWAS data from
either IMPUTE2 (Howie, et al., 2009) or VCF files from the Michigan
and/or Sanger imputation servers. \texttt{gwasurvivr} can also be used
on directly typed data in plink format (\texttt{.bed}, \texttt{.bim} and
\texttt{.fam} files).

Herein, we detail our implementation of the Cox model, generation of the
simulated data and survival benchmarking and graphically report the
correlation of \texttt{gwasurvivr} beta coefficient estimates, minor
allele frequencies (MAF) and p-values with those produced from
SurvivalGWAS\_SV, genipe, and GWASTools.

To reproduce the data and create Figure 1 and Supplementary Figures 2-4,
the data is available on the
\href{https://github.com/suchestoncampbelllab/gwasurvivr_manuscript}{gwasurvivr
manuscript repository}. \href{https://git-lfs.github.com/}{GitHub Large
File Storage (LFS)}.

To clone the whole repository:

\begin{verbatim}
git lfs clone https://github.com/suchestoncampbelllab/gwasurvivr_manuscript.git
\end{verbatim}

\section{Implementation of Survival Model in
gwasurvivr}\label{implementation-of-survival-model-in-gwasurvivr}

\subsection{Modifying coxph}\label{modifying-coxph}

We decrease computation time by decreasing the number of Newton-Raphson
iterations used to optimize the partial likelihood function in the Cox
proportional hazard models. To do this, a survival model was fit using
only non-genetic covariates (i.e.~the SNP is not included and only
covariates are fit); \texttt{survival::coxph} (Therneau and Grambsch,
2000) is modified such that gwasurvivr manually creates the objects
found in the helper function (\texttt{survival::coxph.fit}) that fits
the Cox model.

These variables are then passed to \texttt{survival::coxph.fit}.

\subsection{Benchmarking with survival
package}\label{benchmarking-with-survival-package}

To assess if providing initial estimates from covariates versus using
the survival function as implemented in the survival package improves
computational time, we tested a dataset of 500 individuals at 7255 SNPs
with 1, 2, or 3 covariates. These data are a subset of the simulated
data described in detail below.

The helper function \texttt{gwasurvivr:::coxParam}, adjusted for this
Supplementary documentation is labeled \texttt{gcoxph}. In
\href{https://github.com/suchestoncampbelllab/gwasurvivr_manuscript/blob/master/supplemental_data/code/gcoxph_model.R}{\texttt{gcoxph\_model.R}}
we fit the model without the SNP and the parameter estimates are then
used as initial points for all subsequent models and applied over all
SNPs in the dataset. If there were no covariates, the initial estimates
would be null. The function
\href{https://github.com/suchestoncampbelllab/gwasurvivr_manuscript/blob/master/supplemental_data/code/coxph_model.R}{\texttt{coxph\_model.R}}
implements a \texttt{survival} model (survival package, Therneau and
Grambsch, 2000) without using the optimization starting point obtained
from including covariates in the model.

To test the package runtime over a pre-specified number of iterations
and including 1, 2, or 3 covariates the \texttt{microbenchmark} package
in R was used. The code for
\href{https://github.com/suchestoncampbelllab/gwasurvivr_manuscript/blob/master/supplemental_data/code/SupplementalFigure1.R}{Supplementary
Figure 1}) is available.

By leveraging an initalization point from the analyses with covariates
\texttt{gwasurvivr} (gcoxph) is several seconds faster than the survival
analyses function as implemented in \texttt{survival} (coxph, Therneau
and Grambsch, 2000) in R (\textbf{Supplementary Figure 1}). While this
is a small test dataset, in practice this would be an appreciable
difference when testing across several thousands of samples and millions
of SNPS. In the \texttt{gwasurvivr} package, we opted to use
\texttt{parallel::parApply} instead of \texttt{base::apply} as shown
above to compute across multiple cores.

\section{Computational Experiments}\label{computational-experiments}

We used the University at Buffalo Computational Center for Research (UB
CCR) academic cluster for our benchmarking analyses. Each analysis was
run exclusively on node
\href{https://www.buffalo.edu/ccr/support/research_facilities/general_compute.html}{CPU-L5520}
with the same system specifications, controlling the computational
resources for each run. The UB CCR uses Simple Linux Utility for
Resource Management (SLURM) scheduling for jobs. SLURM scripts to run
the analyses were generated using shell scripts below. Benchmarking was
performed using identical CPU constraints, 1 node (2.27 GHz Clock Rate)
and 8 cores with 24 GB of RAM, on the University at Buffalo Center for
Computational Research supercomputer. With the exception of the larger
sample size tests, these were run using the same node but 12 CPUs.
genipe (Lemieux Perreault, et al., 2016), SurvivalGWAS\_SV (Syed, et
al., 2017), and GWASTools (Gogarten, et al., 2012) were performed as
specified by the authors on available online documentation. We performed
the following benchmarking runtime experiments either against existing
software or against time with varying N and SNP numbers that were
performed:\\
Simulation 1. Compare gwasurvivr against genipe, GWASTools and
SurvivalGWAS\_SV - varying sample sizes (n=100, n=1000, n=5000) and
100,000 SNPs (m=100000) and 3 non-genetic covariates\\
Simulation 2. Comparison of gwasurvivr, genipe, GWASTools and
SurvivalGWAS\_SV with N=5,000 and 100,000 SNPs (m=100,000) with 4
covariates (age, drug treatment, sex and 1 PC), 8 covariates (age, drug
treatment, sex and 5 PCs) and 12 covariates (age, drug treatment, sex
and 9 PCs)\\
Simulation 3. Increasingly larger sample sizes (N=15K, 20K and 25K)
tested on Chromosome 22\\
Simulation 4. Full autosomal GWAS with varying sample sizes (N=3K, 6K
and 9K)

\subsection{Simulating Genotypes and
Phenotypes}\label{simulating-genotypes-and-phenotypes}

\subsubsection{Genotypes}\label{genotypes}

\href{http://mathgen.stats.ox.ac.uk/genetics_software/hapgen/hapgen2.html}{HAPGENv2}
(Su, et al., 2011) was used to generate simulated genetic datasets from
\href{https://mathgen.stats.ox.ac.uk/impute/impute_v1.html\#Using_IMPUTE_with_the_HapMap_Data}{1000
Genomes Project CEU data} (NCBI Build 36) for all benchmarking
experiments. To replicate simulations the 1000 Genomes Project CEU data
should be downloaded in its entirety (only a subset is available on our
GitHub repo). The
\href{https://github.com/suchestoncampbelllab/gwasurvivr_manuscript/tree/master/hapgen2/code}{code}
for all HAPGENv2 simulations are available on our GitHub.

\subsubsection{Phenotypes}\label{phenotypes}

For each sample size tested, survival events (alive/dead) were simulated
as two separate datasets. For the dead dataset, time to event and
covariates were simulated using a normal distribution. For the alive
dataset, time was simulated by randomly sampling weighted probabilities
for times to simulate few samples being censored, covariates were
simulated from a normal distribution.
\href{https://github.com/suchestoncampbelllab/gwasurvivr_manuscript/blob/master/diff_cov_benchmarks/code/add_pcs_covariates.R}{Principal
components (PCs)} were simulated using random normal distributions with
decreasing variance for each additional PC. Furthermore, the
\texttt{.sample} file from IMPUTE2 includes 4 columns (ID\_1, ID\_2,
missing, and sex) which link individuals with their respective
genotypes. For SurvivalGWAS\_SV and GWASTools, the simulated phenotypes
were appended to column 5 onward in the \texttt{.sample} file.

The following genotypes and phenotypes were simulated:\\
\textbf{Simulations 1 and 2.} Subset of chromosome 18 for 100,000 SNPs
1) varying N and 3 covariates done in triplicate and 2) with 4, 8 and 12
covariates

\begin{itemize}
  \item \href{https://github.com/suchestoncampbelllab/gwasurvivr_manuscript/blob/master/hapgen2/code/generate_chr18_subsets_geno.sh}{genotype code}
  \item \href{https://github.com/suchestoncampbelllab/gwasurvivr_manuscript/blob/master/benchmark_experiments/code/simulate_phenotypes_benchmark.R}{phenotype code}  
  \item \href{https://github.com/suchestoncampbelllab/gwasurvivr_manuscript/blob/master/diff_cov_benchmarks/code/add_pcs_covariates.R}{PCs phenotype code}
\end{itemize}

\textbf{Simulation 3.} chromosome 22 (\textasciitilde{}117,000 SNPs) for
larger sample sizes (N=15000-25000)

\begin{itemize}
  \item \href{https://github.com/suchestoncampbelllab/gwasurvivr_manuscript/blob/master/hapgen2/code/generate_largeN_geno.sh}{genotype code}
\end{itemize}

\textbf{Simulation 4.} Full GWAS for N=9000 (the smaller subsets were
just parsed from the data during analyses)

\begin{itemize}
  \item \href{https://github.com/suchestoncampbelllab/gwasurvivr_manuscript/blob/master/hapgen2/code/generate_gwas_geno.sh}{genotype code}   
  \item \href{https://github.com/suchestoncampbelllab/gwasurvivr_manuscript/blob/master/full_gwas_experiments/code/phenotype_sim.R}{phenotype code}
  \item \href{https://github.com/suchestoncampbelllab/gwasurvivr_manuscript/blob/master/full_gwas_experiments/code/simulate_gwas_sample_ids.R}{simulate sample ids code}  
\end{itemize}

\subsection{Benchmarking with other software capable of GWAS coxph
survival
analysis}\label{benchmarking-with-other-software-capable-of-gwas-coxph-survival-analysis}

We benchmarked \texttt{gwasurvivr} with GWAS survival analysis software,
genipe, SurvivalGWAS\_SV and GWASTools using simulated phenotype and
genotype data. Genetic data were formatted as output from IMPUTE2
software (.GEN). Genipe, SurvivalGWAS\_SV, and GWASTools do not directly
take VCF data output from Sanger or Michigan imputation servers.
SurvivalGWAS\_SV does accept VCF files as an input but uncompressed and
not explicity the same format that Sanger and Michigan imputation
servers output, rendering additional steps to be taken. The benchmarking
with IMPUTE2 was done for (1) varying sample sizes and (2) varying
additional non-genetic covariates. Both are described here.

\subsubsection{gwasurvivr}\label{gwasurvivr-1}

The following scripts were used to run gwasurvivr using
\texttt{impute2CoxSurv}. These R scripts are run using a shell script
(SLURM script) that pass the system variables into R (facilitated by the
R package \texttt{batch}).

N=100, 1000 and 5000 with M=100K SNPs + 3 non-genetic covariates in
triplicate:\\
-
\href{https://github.com/suchestoncampbelllab/gwasurvivr_manuscript/blob/master/benchmark_experiments/code/run_gwasurvivr.R}{\texttt{run\_gwasurvivr.R}}\\
-
\href{https://github.com/suchestoncampbelllab/gwasurvivr_manuscript/blob/master/benchmark_experiments/code/create_gwasurvivr_scripts.sh}{\texttt{create\_gwasurvivr\_scripts.sh}}

N=5,000 and M=100K with 4, 8 and 12 covariates:\\
-
\href{https://github.com/suchestoncampbelllab/gwasurvivr_manuscript/blob/master/diff_cov_benchmarks/code/run_gwasurvivr_covs.R}{\texttt{run\_gwasurvivr\_covs.R}}\\
-
\href{https://github.com/suchestoncampbelllab/gwasurvivr_manuscript/blob/master/diff_cov_benchmarks/code/gwasurvivr_cov4.sh}{\texttt{gwasurvivr\_cov4.sh}}\\
-
\href{https://github.com/suchestoncampbelllab/gwasurvivr_manuscript/blob/master/diff_cov_benchmarks/code/gwasurvivr_cov8.sh}{\texttt{gwasurvivr\_cov8.sh}}\\
-
\href{https://github.com/suchestoncampbelllab/gwasurvivr_manuscript/blob/master/diff_cov_benchmarks/code/gwasurvivr_cov12.sh}{\texttt{gwasurvivr\_cov12.sh}}

\subsubsection{genipe}\label{genipe}

For genipe, the shell scripts was used to generate SLURM scripts for
genipe and each sample and SNP set. We used specific settings for
OPENBLAS that are suggested on
\href{http://pgxcentre.github.io/genipe/execution_time.html}{genipe's
website} to ensure that computational efficiency was maximized.

varying sample sizes + 3 non-genetic covariates:\\
-
\href{https://github.com/suchestoncampbelllab/gwasurvivr_manuscript/blob/master/benchmark_experiments/code/create_genipe_scripts.sh}{\texttt{create\_genipe\_scripts.sh}}

additional covariates:\\
-
\href{https://github.com/suchestoncampbelllab/gwasurvivr_manuscript/blob/master/diff_cov_benchmarks/code/genipe_cov4.sh}{\texttt{genipe\_cov4.sh}}\\
-
\href{https://github.com/suchestoncampbelllab/gwasurvivr_manuscript/blob/master/diff_cov_benchmarks/code/genipe_cov8.sh}{\texttt{genipe\_cov8.sh}}\\
-
\href{https://github.com/suchestoncampbelllab/gwasurvivr_manuscript/blob/master/diff_cov_benchmarks/code/genipe_cov12.sh}{\texttt{genipe\_cov12.sh}}

\subsubsection{SurvivalGWAS\_SV}\label{survivalgwas_sv}

To maximize the performance of SurvivalGWAS\_SV, these jobs were run
using ``array'' jobs as recommended by the authors. An
\href{https://www.liverpool.ac.uk/media/livacuk/instituteoftranslationalmedicine/biostats/batchexample.sh}{example
batch script}, provided in the SurvivalGWAS\_SV documentation, was
converted from PBS to SLURM. 24GB of ram was not needed on all runs,
however was used to ensure each run remained uniform. The jobs were
split into array sets of 1000 SNPs for m=100,000, totaling 100 batched
jobs in a single array. We define rate-limiting array as the array index
that had the longest runtime. In the main manuscript, we report
SurvivalGWAS\_SV runtimes as the rate-limiting array runtime. This is an
important caveat and bears consideration when using SurvivalGWAS\_SV.
Depending on availability on the computing cluster, the analyses could
be completed as quickly as the longest individual array job (which is
shown in Figure 1), or potentially the entire runtime could be equal to
the summation runtime of all of the array indices if these cannot be run
simultaneously (or if there are failures with any of the array indices).
The shell script below was used to generate SLURM scripts for
SurvivalGWAS\_SV for each sample and SNP set.

N=100, 1000 and 5000 with M=100K SNPs + 3 non-genetic covariates in
triplicate:\\
-
\href{https://github.com/suchestoncampbelllab/gwasurvivr_manuscript/blob/master/benchmark_experiments/code/create_sv_scripts.sh}{\texttt{create\_sv\_scripts.sh}}

N=5,000 and M=100K with 4, 8 and 12 covariates:\\
-
\href{https://github.com/suchestoncampbelllab/gwasurvivr_manuscript/blob/master/diff_cov_benchmarks/code/sv_cov4.sh}{\texttt{sv\_cov4.sh}}\\
-
\href{https://github.com/suchestoncampbelllab/gwasurvivr_manuscript/blob/master/diff_cov_benchmarks/code/sv_cov8.sh}{\texttt{sv\_cov8.sh}}\\
-
\href{https://github.com/suchestoncampbelllab/gwasurvivr_manuscript/blob/master/diff_cov_benchmarks/code/sv_cov12.sh}{\texttt{sv\_cov12.sh}}

\subsubsection{GWASTools}\label{gwastools}

For GWASTools, the files are converted to GDS format and survival is run
using \texttt{GWASTools::assocCoxPH} within
\href{https://github.com/suchestoncampbelllab/gwasurvivr_manuscript/blob/master/benchmark_experiments/code/gwastools_survival.R}{\texttt{gwastools\_survival.R}}.
The R script was passed to the SLURM scripts using the script
\href{https://github.com/suchestoncampbelllab/gwasurvivr_manuscript/blob/master/benchmark_experiments/code/create_gwastools_scripts.sh}{\texttt{create\_gwastools\_scripts.sh}}.
GWASTools does not run in parallel across multiple cores on a single
computing processor internally, however experienced users could code
this themselves.

N=100, 1000 and 5000 with M=100K SNPs + 3 non-genetic covariates in
triplicate:\\
-
\href{https://github.com/suchestoncampbelllab/gwasurvivr_manuscript/blob/master/benchmark_experiments/code/gwastools_survival.R}{\texttt{gwastools\_survival.R}}\\
-
\href{https://github.com/suchestoncampbelllab/gwasurvivr_manuscript/blob/master/benchmark_experiments/code/create_gwastools_scripts.sh}{\texttt{create\_gwastools\_scripts.sh}}

N=5,000 and M=100K with 4, 8 and 12 covariates:\\
-
\href{https://github.com/suchestoncampbelllab/gwasurvivr_manuscript/blob/master/diff_cov_benchmarks/code/gwastools_survival_covs.R}{\texttt{gwastools\_survival\_covs.R}}\\
-
\href{https://github.com/suchestoncampbelllab/gwasurvivr_manuscript/blob/master/diff_cov_benchmarks/code/gwastools_cov4.sh}{\texttt{gwastools\_cov4.sh}}\\
-
\href{https://github.com/suchestoncampbelllab/gwasurvivr_manuscript/blob/master/diff_cov_benchmarks/code/gwastools_cov8.sh}{\texttt{gwastools\_cov8.sh}}\\
-
\href{https://github.com/suchestoncampbelllab/gwasurvivr_manuscript/blob/master/diff_cov_benchmarks/code/gwastools_cov12.sh}{\texttt{gwastools\_cov12.sh}}

\subsection{Runtime large N chromosomes to test size
limitations}\label{runtime-large-n-chromosomes-to-test-size-limitations}

We tested chr22 with different sample sizes of N=15,000; N=20,000;
N=25,000 using \texttt{gwasurvivr::impute2CoxSurv}. The code for all of
the runs can be found
\href{https://github.com/suchestoncampbelllab/gwasurvivr_manuscript/tree/master/largeN_experiments/code}{here}.
The R script called from the shell scripts to run these analyses is
labeled
\href{https://github.com/suchestoncampbelllab/gwasurvivr_manuscript/tree/master/largeN_experiments/code}{\texttt{run\_bigNs.R}}.

\subsection{Runtime GWAS with different sample
sizes}\label{runtime-gwas-with-different-sample-sizes}

We performed three GWAS (chr1-chr22) with different sample sizes
(n=3000; n=6000; n=9000) using \texttt{gwasurvivr::impute2CoxSurv}. The
code to simulate the GWAS is available on our repository. The R script
used to run these analyses is
\href{https://github.com/suchestoncampbelllab/gwasurvivr_manuscript/tree/master/full_gwas_experiments/code}{\texttt{run\_fullgwas.R}}.
The shell script run these scripts on SLURM can be found
\href{https://github.com/suchestoncampbelllab/gwasurvivr_manuscript/blob/master/full_gwas_experiments/code/submit_fullgwas.sh}{here}.

\section{Time Plots}\label{time-plots}

\subsection{Figure 1}\label{figure-1}

To generate Figure 1 times from the computation runtime were pulled from
SLURM log files and collected using the perl scripts, which can be found
in each of the log folders on our manuscript GitHub repository, compiled
and Figure 1 was generated using the R code shown
\href{https://github.com/suchestoncampbelllab/gwasurvivr_manuscript/blob/master/code/timePlots.R}{here}.

\section{Diagnostic Plots}\label{diagnostic-plots}

Supplementary Figures 2, 3 and 4 below show the correlation of the
coefficient estimates, minor allele frequency and p-values, respectively
between gwasurvivr and all other software assessed. The correlations
show excellent agreement. The R code used to generate supplemental
figures 2-4 can be found
\href{https://github.com/suchestoncampbelllab/gwasurvivr_manuscript/blob/master/supplemental_data/code/SupplementalFigure2-4.R}{here}.

\subsection{Coefficient Estimates}\label{coefficient-estimates}

\subsection{Minor Allele Frequency
(MAF)}\label{minor-allele-frequency-maf}

\subsection{P-value Estimates}\label{p-value-estimates}

\subsection{Full GWAS Runtimes}\label{full-gwas-runtimes}

\section{gwasurvivr calculations}\label{gwasurvivr-calculations}

\subsection{Minor Allele Frequency
(MAF)}\label{minor-allele-frequency-maf-1}

For a given SNP with alleles \(A\) and \(B\), where \(n_{AB}\) and
\(n_{BB}\) are the number of individuals with \(AB\) and \(BB\) genotype
respectively, and \(N\) is the sample size, the expected allele
frequency of allele \(B\) (\(freq_B\)) be can be calculated as:

\[freq_B = \frac{ n_{AB} + 2n_{BB}}{2N}\]

For individual \(i\), the allele dosage of SNP \(j\) (\(D_{ij}\)) with
alleles \(A\) and \(B\), where allele \(B\) is the effect allele and
\(p_{AB}\) and \(p_{BB}\) are the posterior genotype probabilities as
computed by the imputation, is calculated as:

\[ D_{ij} = p_{AB_{ij}} + 2 \cdot p_{BB_{ij}} \]

For SNP \(j\) The estimated allele frequency of an effect allele \(B\)
(\(\theta_{B_j}\)) can therefore be calculated as:

\[\theta_{B_j} = \frac{\sum_{i=1}^{N} D_{ij} }{2N}\]

This was coded in R as follows:

\begin{Shaded}
\begin{Highlighting}[]
\CommentTok{# calculate MAF}
\CommentTok{# genotypes variable is a matrix of dosages, }
\NormalTok{## where each column is a sample and each row is a SNP}
\NormalTok{exp_freq_A1 <-}\StringTok{ }\KeywordTok{round}\NormalTok{(matrixStats}\OperatorTok{::}\KeywordTok{rowMeans2}\NormalTok{(genotypes)}\OperatorTok{*}\FloatTok{0.5}\NormalTok{,}\DecValTok{4}\NormalTok{)}
\NormalTok{MAF <-}\StringTok{ }\KeywordTok{ifelse}\NormalTok{(exp_freq_A1 }\OperatorTok{>}\StringTok{ }\FloatTok{0.5}\NormalTok{,}
                  \DecValTok{1}\OperatorTok{-}\NormalTok{exp_freq_A1,}
\NormalTok{                  exp_freq_A1)}
\end{Highlighting}
\end{Shaded}

\subsection{Imputation quality metric}\label{imputation-quality-metric}

\subsubsection{Michigan Imputation
Server}\label{michigan-imputation-server}

For the Michigan imputation server, imputation is performed using the
minimac3 algorithm (Das et al., 2016). minimac3 computes and outputs an
imputation quality metric known as \(R^2\). \(R^2\) is the estimated
value of the squared correlation betwen imputed genotypes and true,
unobserved genotypes (Das et al, 2016). The \(R^2\) value is extracted
directly from the Michigan imputation output VCF in
\texttt{gwasurvivr::michiganCoxSurv}

\subsubsection{Sanger Imputation Server}\label{sanger-imputation-server}

For the Sanger imputation server, we grab the \texttt{INFO} field
directly from the VCF file in \texttt{gwasurvivr::sangerCoxSurv}. The
INFO field is the IMPUTE2 (Howie, et al., 2009) score as calculated by
the \texttt{bcftools\ +\ impute-info} plugin from posterior genotype
probabilities (McCarthy et al., 2016).

\subsubsection{IMPUTE2 Imputation}\label{impute2-imputation}

The INFO score for IMPUTE2 (Howie, et al., 2009) results are not
calculated in \texttt{gwasurvivr} internally, instead we use the INFO
scores that are provided in a separate file after performing imputation
(\texttt{.info} file). Users select SNPs from the \texttt{.info} file to
remove based on preferred criterion (ie INFO \textless{} .8) these are
then used in the argument \texttt{exclude.snps} in
\texttt{impute2CoxSurv} to filter out the SNPs prior to analysis.

\FloatBarrier

\newpage

\pagestyle{plain} \fancyhead[L]{} \fancyhead[R]{}
\fancyfoot[C]{\thepage}

\chapter{Acute Lymphobastic Leukemia (ALL) GWAS}

\FloatBarrier

\newpage

\pagestyle{plain} \fancyhead[L]{} \fancyhead[R]{}
\fancyfoot[C]{\thepage}

\chapter{Conclusion}

\FloatBarrier
\newpage
\pagestyle{plain} \fancyhead[L]{} \fancyhead[R]{}
\fancyfoot[C]{\thepage}

\chapter*{REFERENCES}

\singlespacing
\setlength{\parindent}{-0.5in} \setlength{\leftskip}{0.4in}
\setlength{\parskip}{6pt} \noindent

\hypertarget{refs}{}
\hypertarget{ref-Rizvi_2018}{}
Rizvi, Abbas A, Ezgi Karaesmen, Martin Morgan, Leah Preus, Junke Wang,
Michael Sovic, and Lara Sucheston-Campbell. 2018. ``Gwasurvivr: An R
Package for Genome Wide Survival Analysis.'' \emph{bioRxiv}, May. Cold
Spring Harbor Laboratory.
doi:\href{https://doi.org/10.1101/326033}{10.1101/326033}.


\end{document}
